\documentclass{../GPUSPHtemplate}

\title{GPUSPH Validation basis}

\author{}

\date{\currentver\ --- November 2018}

\begin{document}

\maketitle
\tableofcontents
\clearpage
\section{Introduction}

In the present document the various validation cases tested to check GPUSPH works as expected
are described, together with their setup in the user interface.

\section{Spheric2}


\subsection{Case Study Setup and Details}
%\vspace*{2pt}
   
\begin{itemize}
\item \textbf{Description of the Problem}: This case is based on the experimental setup of Kleefsman et al. \citep{Kleefsman}
  in their work on modeling hydraulics problems with wave impact.
  In fact, this case study has become a classic validation problem in the literature and has been reiterated multiple times,
  as can be seen in Issa and Violeau's paper \citep{SPHERIC} and in Leroy's thesis \citep{AgnesLeroy}. This case study is well
  suited for the study of strong surface deformation capabilities on an algorithm. The problem entails a model dam break
  problem with a model rigid obstacle upstream of the dam's collapse. There is no analytical solution to a problem of this type,
  so the numerical results will be compared to the measurements done by Kleefsman et al. \citep{Kleefsman}.
  Since the mathematical setup of the problem is out of the scope of this report, most variables and parameters are not non-dimensionalized except for: 
  \begin{equation}
    \begin{array}{l}
      \displaystyle{  t^{+} \doteq \frac{t}{\sqrt{L/g}} } \medskip \\ 
    \end{array}
    \end{equation}  

  \begin{figure}[h!]
    \includegraphics[scale =0.5]{figures/SPHERIC_GEOMETRY}
    \centering
    \caption{ The 3-D geometric details of the Spheric2 case can be seen in this figure. The dimensions and the locations of the height probes can be seen here. The numbering of the fluid height sensors is from H1 to H4, as can be seen above the vertical lines in red indicating the location of the sensors.}
    \label{fig:SphericGeometry}
  \end{figure}
  
  \begin{figure}[h!]
    \includegraphics[scale =0.6]{figures/SPHERIC_NODES}
    \centering
    \caption{ The geometric details of the obstacle used in Spheric2 can be seen in this figure. The locations of the pressure probes can be seen here. The numbering of the pressure sensors is from P1 to P8, as can be seen to the left of the crosses in red indicating the location of the sensors.}
    \label{fig:Spheric_Nodes}
  \end{figure}
  
\item \textbf{Problem Setup}: This case involves the study of the deviation of numerical results from the experimental results available.
  Following Kleefsman's work, we setup numerical pressure sensors along the rigid obstacle, as can be seen in the figure \ref{fig:Spheric_Nodes}.
  We also place numerical wave height gauges, whose locations can be seen in the figure \ref{fig:SphericGeometry}.
  The geometric details can be seen in the figure \ref{fig:SphericGeometry}. Note that figures \ref{fig:SphericGeometry},
  \ref{fig:Spheric_Nodes} for this test case have been imported from Kleefsman et al. \citep{Kleefsman}.
  The experimental results contain data on the pressure variation in time along the sensors seen in figure \ref{fig:Spheric_Nodes}
  and data on the fluid height along the gauges shown in figure \ref{fig:SphericGeometry}. The analysis on the results of this
  test case will be based on a visual comparison of the numerical versus the experimental results.    
  
\item \textbf{The Numerical and Physical Parameters Studied:} The tests run for this case are designed to study the performance
  of the free surface deformation capabilities of the solver. Among the many parameters varied in the simulations,
  three important parameters stand out. The behavior of the solver strongly changes with kinematic viscosity,
  particle size and the turning on of the turbulence model. As will be seen in the results section,
  the solver generates less noise when the turbulent model is turned on
  for models with significantly high Reynolds numbers\footnote{A detailed study of when the turbulent model is
    recommended to be used is out of the scope of this report.}. The simulations have a fluid particle number
  of about 3 million particles. It is important to note that our numerical
  setups do not model a physical gate, and thus the dam's fluid falls instantaneouly, and this is not the case in
  Kleefsman et al.'s work. The following presents the parameters used and the tests run for this test case are presented below:   
  \begin{itemize}
  \item Numerical Speed of Sound: ${c_n} = 40 \; m/s$
  \item Smoothing Length to Particle Size Ratio: $\dfrac{h}{\delta r}=1.3$ 
  \item The Gravity Field is: $g = -9.81.0 \; {m}/{s^2}$
  \item The Density is: $\rho = 1000 \; {kg}/{m^3} $
  \item The Length Scale is: $L = 0.55 \; m$
  \item Simulation Time\footnote{Test 6 was run for $t^+=34$.} $t^+ = 26 $
  \item Particle size: $0.01m$
  \item Kinematic viscosity: $\nu = 1 \times 10^{-6} (m^2/s)$
  \item Whatever the density diffusion formulation, a density diffusion coefficient of 0.1
  \item With semi-analytical boundaries, a density equation based on the summation definition of the density
  \end{itemize}
  
  \begin{table}[h!]
    \centering
    \begin{tabular}{ |p{3cm}||p{2cm}|p{2cm}|p{2cm}|p{3cm}|  } 
      \hline
      \multicolumn{5}{|c|}{Numerical Tests Studied} \\
      \hline
      Numerical Test & Boundary formulation & Turbulent Model & Density diffusion \\
      \hline
      Test 1 & Lennard-Jones      & none & Colagrossi \\
      Test 2 & Lennard-Jones      & SPS & Colagrossi \\
      Test 3 & Dynamic (3 layers) & SPS  & Colagrossi \\
      Test 4 & Semi-analytical    & $k-\epsilon$ & Brezzi
      \hline
    \end{tabular}
  \end{table}

\vspace{20pt}
   
\end{itemize}

\subsection{Results}

The results of this case are compared with either the experimental results from Kleefsman et al. \citep{Kleefsman} and with the numerical results of Leroy \citep{AgnesLeroy}, where applicable. The results of the tests are in the form of pressure values recorded in time at the different sensors and in values of fluid height at the different height sensors in time. Figure (\ref{fig:Crash}) shows snapshots at the time $t^+=2.95$ of different simulations. The image at the top left of figure (\ref{fig:Crash}) is of test 1, the top right is representative of the coarse tests (2,3,4,5)\footnote{Presenting snapshots of all the tests of similar coarseness is tedious and redundant.} and the image at the bottom, the refined test 6\footnote{For the definition of tests, refer to table 5 at the end of section 2.3.1.}. The snapshot of test 1 shows particles stuck to the basin walls, as was reported in section 5.2 of Leroy's thesis \citep{AgnesLeroy}, and this is due to the high viscosity of the fluid in the system. Test 1 will be the only study compared with the Leroy's results. The snapshot of the coarse simulations presents a significant difference in fluid crashing behavior when compared to the image of the finer simulation seen below it. A possible explanation for this discrepancy is that the finer the fluid particles in the model are, the more they approximate the mass of the physical fluid, and thus the more the energy of the numerical model approximates the physical system is. Theoretically, the total mass of the physical fluid in this model should be $m= 675.4 $ kg, whereas, the fluid used in test 6 has a total mass of $m = 670.890$ kg and the the fluid used in the rest of the tests, a total mass of $m=658.735$ kg. Since the finer simulation uses a larger total mass, and by extension the energy, the finer simulation is expected to have more pronounced crashing behavior, as is seen in the snapshots. Test 6 also shows better performance when the results from the testpoints are compared to the results of experiments (2,3,4,5). 

%%%%%%%%%% CRASH
\begin{figure}[H]

	\begin{subfigure}[b]{0.7\linewidth}
	 \centering
	\hspace*{3.3cm} \includegraphics[scale=0.45]{figures/ScaleCrash.png}
	 \end{subfigure}	
	
  \begin{subfigure}[b]{0.75\linewidth}
    \centering
     \hspace*{-5cm} \includegraphics[width=0.75\linewidth]{figures/ViscousCrash3.png} 
  \end{subfigure}%%
  \begin{subfigure}[b]{0.75\linewidth}
    \centering
    \hspace*{-10cm} \includegraphics[width=0.75\linewidth]{figures/CoarseCrash3.png} 
  \end{subfigure} 
	\vspace*{2pt}
	
	\begin{subfigure}[b]{1\linewidth}
	\centering
	\includegraphics[width=1\linewidth]{figures/RefinedCrash3.png}
	\end{subfigure}
	
	\caption{The images shown above represent snapshots at $t^+ = 2.95 $ for three different simulations. The viscous and coarse test, $\delta r = 0.017651 \; m, \nu = 0.01 \; m^2/s$, is seen at the top left, the coarse type simulation, $\delta r = 0.017651 \; m, \nu = 1 \times 10^{-6} \; m^2/s$, is seen at the top right and the refined simulation from test 6, $\delta r = 0.0056 \; m, \nu = 1 \times 10^{-6} \; m^2/s$ is seen at the bottom center. The velocities of the three snapshots follow the color scale  seen above.}
	\label{fig:Crash}
\end{figure}
%%%%%%%%%%%%%%%%%%%

%%%%%%%%%%%Experiments 2 to 5
\begin{figure}[H]
  	 \centering 	
	 \begin{subfigure}[b]{1\linewidth}
	    \includegraphics[width=1\linewidth]{figures/DATAforREPORT/SPHERIC2SA/Experiment1/C_P2_P5_H2_H4.png} 
	    \caption{Test 1}
	 \end{subfigure}%% 
  
  	 \begin{subfigure}[b]{1\linewidth}
    		\includegraphics[width=1\linewidth]{figures/DATAforREPORT/SPHERIC2SA/Experiment2/B_P2_P5_H2_H4.png} 		
		\caption{Test 2}
 	 \end{subfigure}   
\end{figure}

\begin{figure}[H]\ContinuedFloat
  	\begin{subfigure}[b]{1\linewidth}
    		 \includegraphics[width=1\linewidth]{figures/DATAforREPORT/SPHERIC2SA/Experiment3/B_P2_P5_H2_H4.png} 
    		 \caption{Test 3}
  	\end{subfigure}%%
  	
  	\begin{subfigure}[b]{1\linewidth}
    		\includegraphics[width=1\linewidth]{figures/DATAforREPORT/SPHERIC2SA/Experiment4/B_P2_P5_H2_H4.png} 
		\caption{Test 4}  	
  	\end{subfigure}
  	
\end{figure}  	 
  
\begin{figure}[H]\ContinuedFloat  
  	\begin{subfigure}[b]{1\linewidth}
   		\includegraphics[width=1\linewidth]{figures/DATAforREPORT/SPHERIC2SA/Experiment5/B_P2_P5_H2_H4.png} 
		\caption{Test 5}
  	\end{subfigure}%% 

     \caption{This figure lists the results obtained for test 1 to 5, with the results for each test paired together. Listing all the data is tedious and redundant. The data from P2 and P5, are on the top 2 sub-plots, and the data from H2 and H4 on the bottom 2 sub-plots of each sub-figure denoted by (a,b,c,d,e). The red plots display the data as obtained from GPUSPH. No FV data exists for the height sensors in (a), and WCSPH and ISPH refer to weakly compressible and incompressible SPH.}
\label{fig:TestpointsAll}
\end{figure}
%%%%%%%%%%%%%%%%%

In analyzing the data out of the testpoints of test 1, we refer to figure (\ref{fig:TestpointsAll}a). WCSPH and ISPH refer, respectively, to the two SPH models Leroy studied, in which she performed simulations using a weakly compressible and an incompressible SPH model \citep{AgnesLeroy}. As is seen in the plots, GPUSPH\footnote{As mentioned previously, GPUSPH is a WCSPH solver.} largely outperforms the WCSPH model used by Leroy. GPUSPH has significantly lower levels of noise when compared to the previous WCSPH models used. However, GPUSPH does not significantly outperform the ISPH model (except by not realizing negative pressures) used by Leroy, when their data is compared to the FV results seen on the pressure testpoint plots. \\ 

In analyzing the tests compared against Kleefsman et al.'s experiments, it is important to note that simulations that were run with the $k$--$\epsilon$ turbulent model turned on, generally faired better than their counterparts with the model turned off. This is seen in the significant reduction of noise in the simulation results between tests 2 and 3, and between tests 4 and 5, in which tests 2 and 4 faired better than their counterparts. Next, we look at the differences between the simulations with different Ferrari coefficient values $f$. Between tests 2 and 4 (with $k$--$\epsilon$), we note that test 4's use of $f = 0.1$ significantly reduces the solver's generation of large pressure peaks. A possible explanation to this reduction in pressure peaks is because the Ferrari coefficient adds more weight to the density correcting terms used in the algorithm (see section 1.2.6); therefore reducing errors in density and in turn pressure.

\begin{figure}[H]
	\centering
  \begin{subfigure}[b]{1\linewidth}
      \includegraphics[width=1\linewidth]{figures/DATAforREPORT/SPHERIC2SA/Experiment6/B_P1toP4.png}
      \caption{P1 to P4}
  \end{subfigure}%%
  
  \begin{subfigure}[b]{1\linewidth}
   		\includegraphics[width=1\linewidth]{figures/DATAforREPORT/SPHERIC2SA/Experiment6/B_P5toP8.png} 
		\caption{P5 to P8}  
  \end{subfigure} 
  
\end{figure}

\begin{figure}[H]\ContinuedFloat    

  \begin{subfigure}[b]{1\linewidth}
   		\includegraphics[width=1\linewidth]{figures/DATAforREPORT/SPHERIC2SA/Experiment6/B_WaveGauges.png} 
 		\caption{H1 to H4} 
  \end{subfigure} 
  
  \caption{This figure lists the results obtained for test 6, with the results paired as indicated above. The complete set of results for this test are displayed because of the strong agreement between the simulation and Kleefsman's results. The set of parameters used in test 6 are the most effective out of all the non-viscous experiments, and thus a complete display of the data is warranted.}

\end{figure}


Referring to test 6, which uses both the $k$--$\epsilon$ and $f=0.1$, we note that the data admits significantly better results than those of tests 2 to 5. Most of the unwanted large pressure peaks are no longer resolved except in testpoint P8. Most of the noise is also cleared away. However, as is the case with some tests in GPUSPH, negative pressures can often erroneously quantify\footnote{This is due to Cole's equation of state's \ref{eq:normal compressibility} sensitivity to density fluctuations.}, and this is issue is not cleared in this test, as is seen in testpoints P3, P5, P6, P7 and P8. The water height sensors also reveal test 6's agreement with Kleefsman et al.'s data. However, noise is apparent in sensor H1 and H2, and that is due to the violent crashing that occurs when the fluid impacts the object upstream of it. Parts of the fluid are projected violently and the height sensors are thus perturbed with sensory peaks, as is seen in sensors H1 and H2. The gaps seen in the height data correspond to instances where gaps between the fluid occur as the fluid responds to the impact. The gaps do not significantly disturb the flow of data. It is important to note that the data in all of the sensors admits a small phase lag. It is not clear why this occurs, however a future test studying the fluid's response change as a result of a non-instantaneous gate release will be beneficial\footnote{It is important that the future modeler model the gate release based on the gate release parameters used in Kleefsman et al.'s work.}.  

\section{Hydrostatic basin}

\subsection{Case Study Setup and Details}
\vspace*{2pt}
\begin{itemize}
\item \textbf{Description of the Problem}: We consider a simple 3-dimensional rectangular basin which is
  filled with fluid particles, and subjected to a gravity-like field force acting in a parallel and
  opposite manner with respect to the bottom's normal. Figure (\ref{fig:HydrostaticBasin}) represents the
  geometric setup of this case. The problem is non-dimensionalized by using the system's scale length $L$
  and other physical parameters. The non-dimensional parameters and variables are as seen below:   
  \begin{equation}
    \begin{array}{l}
      \displaystyle{  z^{+} \doteq \frac{z}{L} } \medskip \\ 
      \displaystyle{  t^{+} \doteq \frac{t}{\sqrt{L/g}} } \medskip \\ 
      \displaystyle{  P^{+} \doteq \frac{P}{\rho gL} } \medskip \\ 
      \displaystyle{  F^{+} \doteq \frac{F}{\rho g} } 
    \end{array}
  \end{equation}  
  
\item \textbf{Case Study}:  This case involves the study of the deviation of numerical results
  from the analytical solution. The solution is based on the hydrostatic equation which can be
  derived from the RANS equation projected along the vertical z-axis: 
  
  \begin{equation}\label{eq:HydroDifferential}
    0 = F^{+}+ \frac{d P^+}{d z^+}
  \end{equation}
  
  with the differential equation being an ordinary differential equation presented as a boundary value problem.
  The wall boundaries are solid and the velocity is nullified. The resulting pressure profiles
  depend on the gravity-like force field. The chosen parameters and
  the solution to equation (\ref{eq:HydroDifferential}) are presented below.
  
  The forcing function is chosen to be: $ F^+ = \text{C}  $  
    
  \begin{equation}\label{eq:HydroStatic}
    P^+(z^+) =  -F^{+} z^+  + C
  \end{equation}
  
  The constant in equations (\ref{eq:HydroStatic}) is defined as:\\
  $C = P^+(z^+ = 0) = 1$
  
\item \textbf{The Numerical and Physical Parameters Studied:} This study sets up three tests.
  We look to study the stability of hydrostatic iterations in GPUSPH with three boundary formulations available in GPUSPH:
  Lennard-Jones, Dynamic boundaries and Semi-analytical boundaries.
  The particle size of this problem is related to particle number in the problem through the following representation,
  where \textit{N} is the particle number per unit
  \textit{L}\footnote{The basin's boundaries slightly reduce the expected number of particles (per unit $L$)
    along the \textit{x} and \textit{y} axes due to the fluid being constrained from multiple sides,
    as opposed to the number of particles along \textit{z}. }:
  \begin{equation}
    \delta r = \frac{L}{N} 
  \end{equation}            
  
  \begin{itemize}
  \item Numerical Speed of Sound: ${c_n} = 40 \; m/s$
  \item Smoothing Length to Particle Size Ratio: $\dfrac{h}{\delta r}=1.3$ 
  \item Kinematic Viscosity\footnote{To be used in the numerical setup only}: $\nu = 0.05 \; {m^2}/{s}$
  \item The Gravity Field is: $g = -1.0 \; {m}/{s^2}$
  \item The Density is: $\rho = 1.0 \; {kg}/{m^3} $
  \item The Length Scale is: $L = 1.0 \; m$
  \item Simulation Time: $t^+ = 600 $ 
  \end{itemize}
  
\end{itemize}

\subsection{Results}
Modeling hydrostatic phenomenon is not trivial in SPH due to the tendency of the particles
to reorder themselves at every temporal iteration, and assign to themselves small random velocities;
thus perturbing the static hypothesis required for resolving hydrostatic systems. However, with accurate
mathematical models in SPH and efficient algorithms, hydrostatic phenomena can be effectively modeled.\\
%Before GPUSPH incorporated the SA solid boundary model, hydrostatic systems almost always had issues with particle stability near a basin's wall. \\

The results of this case show GPUSPH's capacity at resolving hydrostatic phenomena. The results obtained show
both the pressure profiles at a their initial and final conditions, and the L$_2$ error's progression in time.
The L$_2$ error\footnote{The L$_2$ error used in this case is defined by equation (\ref{L2Error}), but with
  the scalar pressure field in place of the vector velocity field.} is important as it shows whether
GPUSPH displays converging numerical results for hydrostatic phenomena. Tests 1 and 2 are shown in
figures (\ref{fig:HydroExp1}), (\ref{fig:HydroExp1L2}) and (\ref{fig:HydroExp2}), (\ref{fig:HydroExp2L2})
respectively. As can be seen from figure (\ref{fig:HydroExp1}) and (\ref{fig:HydroExp2}) the pressure
profiles at the initial state $t^+=0$ and the final state $t^+=600$ are displayed. The initial state of
the particles in the basin is initialized with a hydrostatic profile, where the particles are arranged
in a Cartesian manner\footnote{As our figures are two dimensional, they do not explicity show the total
  number of particles in the basin, since most of the particles exist initially alongside each other on flat planes.
  At later time steps however, the particles rearrange and are found at different positions, as is evident
  by the plots at the final iteration.} at every $\delta r$. Test 1's results at the final time step show
a deviation of the particles' pressure distribution away from the reference distribution. The results of
test 1's final iteration show that there is an almost constant deviation, w.r.t the z-axis, in the
pressure values of the particles. The particles also show a small drift in their vertical positions,
where the particles move slightly below the initially defined surface at $z^+=1$, and they move closer
to the basin floor, moving below the initially lowest particle layer at $z^+ = 0.1$. Although test 1's
results at the final iteration appear to be significantly erroneous, the L$_2$ error in time seen in
figure (\ref{fig:HydroExp1L2}) reveals that the solution is in a convergent trend, because of the
negative slopes and stable behavior of the error. This brings into context GPUSPH's current capacity
at simulating stable hydrostatic phenomena, and looking at test 2, the results also reveal positive results.
Test 2 in figure (\ref{fig:HydroExp2}b) shows that as is seen in test 1, the particles exhibit a deviation
in their pressure distribution, and they reveal a small drift in position. The drift in vertical position
is also away from the initially defined surface at $z^+=0$ and under the initially lowest particle layer
at $z^+=0.05$. To quantify how these deviations effect the convergence of model, the L$_2$ error, in time,
is analyzed from figure (\ref{fig:HydroExp2L2}). Although initially the L$_2$ error in test 2 is $ >50\% $
of the initial L$_2$ error in test 1, the slope, w.r.t time, of the L$_2$ in test 2 drops more rapidly
than in test 1. This results in an L$_2$ error at $t^+=600$ to be $\approx 50\%$ of the respective L$_2$
error in test 1. The results of test 3 did not show any effective improvement from the results of test 2;
therefore its results are not displayed, and the \textit{nlexpansion} factor chosen does not significantly alter the solver's results.\\ 

The analysis of the results of this case study show that GPUSPH is effectively capable of simulating
hydrostatic phenomena since the solvers show convergent behavior. The convergent behavior is evident
for both tests and is more pronounced for models with finer particle distributions, as seen in test 2.
However, more studies are to be performed in order to study under which parameter settings do
GPUSPH's solvers lose the good convergence shown here. This is required because it is important
to know what and where the limitations of a solver are. 
\vfill
  

\begin{figure}[H]
             
  \begin{subfigure}{.9\textwidth}
    \centering
    \includegraphics[width=.9\linewidth]{figures/DATAforREPORT/Hydrostatic/10Particles/Pressure_10particles_Initial.png}
    \vspace*{5pt}
    \caption{Initial State}
  \end{subfigure}%
  
  \begin{subfigure}{.9\textwidth}
    \centering 
    \includegraphics[width=.9\linewidth]{figures/DATAforREPORT/Hydrostatic/10Particles/Pressure_10particles_Final.png}
    \caption{Final State}
  \end{subfigure}
  
  \caption{The images seen above present the numerical and theoretical pressure profiles along the basin's height. The initial profile at $t^+=0$ is seen at the top and with the final pressure profile at $t^+=600$ in the image just under. The results are for test 1 of section 2.2.1. It is important to note that the theoretical profile is based on the initial distribution of the particles and thus extends from the particles at the surface of the basin at $z^+=1$ to the bottom particle's postion at a distance $\delta r = 0.1 $ from the floor at $z^+=0$. }
  \label{fig:HydroExp1}
\end{figure}

\begin{figure}[H]
  
  \begin{subfigure}{.9\textwidth}
    \centering
    \includegraphics[width=.9\linewidth]{figures/DATAforREPORT/Hydrostatic/20Particles/Pressure_20particles_Initial.png}
    \vspace*{5pt}
    \caption{Initial State}
  \end{subfigure}%
  
  \begin{subfigure}{.9\textwidth}
    \centering 
    \includegraphics[width=.9\linewidth]{figures/DATAforREPORT/Hydrostatic/20Particles/Pressure_20particles_Final.png}
    \caption{Final State}        
  \end{subfigure}
  
  \caption{The images seen above present the numerical and theoretical pressure profiles along the basin's height. The initial profile at $t^+=0$ is seen at the top and with the final pressure profile at $t^+=600$ in the image just under. The results are for test 2 of section 2.2.1. It is important to note that the theoretical profile is based on the initial distribution of the particles and thus extends from the particles at the surface of the basin at $z^+=1$ to the bottom particle's postion at a distance $\delta r = 0.05 $ from the floor at $z^+=0$. }
  
  \label{fig:HydroExp2}
  
\end{figure}

\begin{figure}[H]
  
  \centering
  \includegraphics[scale=0.6]{figures/DATAforREPORT/Hydrostatic/10Particles/Hydrostatic_L2_10particles.png}
  \caption{The plot seen above represents the L$_2$ error as it progresses in time for test 1 of section 2.2.1, where we note the decreasing trend of the error in time. The data is presented as a semi-log plot. }
  \label{fig:HydroExp1L2}	
\end{figure}

\begin{figure}[H]
  
  \centering
  \includegraphics[scale=0.6]{figures/DATAforREPORT/Hydrostatic/20Particles/L2_Hydrostatic20Part.png}
  \caption{The plot seen above represents the L$_2$ error as it progresses in time for test 2 of section 2.2.1, where we note the decreasing trend of the error in time. The data is presented as a semi-log plot. }
  \label{fig:HydroExp2L2}
\end{figure}


\bibliography{../gpusph-manual.bib}

\end{document}

